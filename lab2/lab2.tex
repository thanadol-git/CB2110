% Options for packages loaded elsewhere
\PassOptionsToPackage{unicode}{hyperref}
\PassOptionsToPackage{hyphens}{url}
\PassOptionsToPackage{dvipsnames,svgnames,x11names}{xcolor}
%
\documentclass[
  letterpaper,
  DIV=11,
  numbers=noendperiod]{scrartcl}

\usepackage{amsmath,amssymb}
\usepackage{iftex}
\ifPDFTeX
  \usepackage[T1]{fontenc}
  \usepackage[utf8]{inputenc}
  \usepackage{textcomp} % provide euro and other symbols
\else % if luatex or xetex
  \usepackage{unicode-math}
  \defaultfontfeatures{Scale=MatchLowercase}
  \defaultfontfeatures[\rmfamily]{Ligatures=TeX,Scale=1}
\fi
\usepackage{lmodern}
\ifPDFTeX\else  
    % xetex/luatex font selection
\fi
% Use upquote if available, for straight quotes in verbatim environments
\IfFileExists{upquote.sty}{\usepackage{upquote}}{}
\IfFileExists{microtype.sty}{% use microtype if available
  \usepackage[]{microtype}
  \UseMicrotypeSet[protrusion]{basicmath} % disable protrusion for tt fonts
}{}
\makeatletter
\@ifundefined{KOMAClassName}{% if non-KOMA class
  \IfFileExists{parskip.sty}{%
    \usepackage{parskip}
  }{% else
    \setlength{\parindent}{0pt}
    \setlength{\parskip}{6pt plus 2pt minus 1pt}}
}{% if KOMA class
  \KOMAoptions{parskip=half}}
\makeatother
\usepackage{xcolor}
\setlength{\emergencystretch}{3em} % prevent overfull lines
\setcounter{secnumdepth}{-\maxdimen} % remove section numbering
% Make \paragraph and \subparagraph free-standing
\makeatletter
\ifx\paragraph\undefined\else
  \let\oldparagraph\paragraph
  \renewcommand{\paragraph}{
    \@ifstar
      \xxxParagraphStar
      \xxxParagraphNoStar
  }
  \newcommand{\xxxParagraphStar}[1]{\oldparagraph*{#1}\mbox{}}
  \newcommand{\xxxParagraphNoStar}[1]{\oldparagraph{#1}\mbox{}}
\fi
\ifx\subparagraph\undefined\else
  \let\oldsubparagraph\subparagraph
  \renewcommand{\subparagraph}{
    \@ifstar
      \xxxSubParagraphStar
      \xxxSubParagraphNoStar
  }
  \newcommand{\xxxSubParagraphStar}[1]{\oldsubparagraph*{#1}\mbox{}}
  \newcommand{\xxxSubParagraphNoStar}[1]{\oldsubparagraph{#1}\mbox{}}
\fi
\makeatother

\usepackage{color}
\usepackage{fancyvrb}
\newcommand{\VerbBar}{|}
\newcommand{\VERB}{\Verb[commandchars=\\\{\}]}
\DefineVerbatimEnvironment{Highlighting}{Verbatim}{commandchars=\\\{\}}
% Add ',fontsize=\small' for more characters per line
\usepackage{framed}
\definecolor{shadecolor}{RGB}{241,243,245}
\newenvironment{Shaded}{\begin{snugshade}}{\end{snugshade}}
\newcommand{\AlertTok}[1]{\textcolor[rgb]{0.68,0.00,0.00}{#1}}
\newcommand{\AnnotationTok}[1]{\textcolor[rgb]{0.37,0.37,0.37}{#1}}
\newcommand{\AttributeTok}[1]{\textcolor[rgb]{0.40,0.45,0.13}{#1}}
\newcommand{\BaseNTok}[1]{\textcolor[rgb]{0.68,0.00,0.00}{#1}}
\newcommand{\BuiltInTok}[1]{\textcolor[rgb]{0.00,0.23,0.31}{#1}}
\newcommand{\CharTok}[1]{\textcolor[rgb]{0.13,0.47,0.30}{#1}}
\newcommand{\CommentTok}[1]{\textcolor[rgb]{0.37,0.37,0.37}{#1}}
\newcommand{\CommentVarTok}[1]{\textcolor[rgb]{0.37,0.37,0.37}{\textit{#1}}}
\newcommand{\ConstantTok}[1]{\textcolor[rgb]{0.56,0.35,0.01}{#1}}
\newcommand{\ControlFlowTok}[1]{\textcolor[rgb]{0.00,0.23,0.31}{\textbf{#1}}}
\newcommand{\DataTypeTok}[1]{\textcolor[rgb]{0.68,0.00,0.00}{#1}}
\newcommand{\DecValTok}[1]{\textcolor[rgb]{0.68,0.00,0.00}{#1}}
\newcommand{\DocumentationTok}[1]{\textcolor[rgb]{0.37,0.37,0.37}{\textit{#1}}}
\newcommand{\ErrorTok}[1]{\textcolor[rgb]{0.68,0.00,0.00}{#1}}
\newcommand{\ExtensionTok}[1]{\textcolor[rgb]{0.00,0.23,0.31}{#1}}
\newcommand{\FloatTok}[1]{\textcolor[rgb]{0.68,0.00,0.00}{#1}}
\newcommand{\FunctionTok}[1]{\textcolor[rgb]{0.28,0.35,0.67}{#1}}
\newcommand{\ImportTok}[1]{\textcolor[rgb]{0.00,0.46,0.62}{#1}}
\newcommand{\InformationTok}[1]{\textcolor[rgb]{0.37,0.37,0.37}{#1}}
\newcommand{\KeywordTok}[1]{\textcolor[rgb]{0.00,0.23,0.31}{\textbf{#1}}}
\newcommand{\NormalTok}[1]{\textcolor[rgb]{0.00,0.23,0.31}{#1}}
\newcommand{\OperatorTok}[1]{\textcolor[rgb]{0.37,0.37,0.37}{#1}}
\newcommand{\OtherTok}[1]{\textcolor[rgb]{0.00,0.23,0.31}{#1}}
\newcommand{\PreprocessorTok}[1]{\textcolor[rgb]{0.68,0.00,0.00}{#1}}
\newcommand{\RegionMarkerTok}[1]{\textcolor[rgb]{0.00,0.23,0.31}{#1}}
\newcommand{\SpecialCharTok}[1]{\textcolor[rgb]{0.37,0.37,0.37}{#1}}
\newcommand{\SpecialStringTok}[1]{\textcolor[rgb]{0.13,0.47,0.30}{#1}}
\newcommand{\StringTok}[1]{\textcolor[rgb]{0.13,0.47,0.30}{#1}}
\newcommand{\VariableTok}[1]{\textcolor[rgb]{0.07,0.07,0.07}{#1}}
\newcommand{\VerbatimStringTok}[1]{\textcolor[rgb]{0.13,0.47,0.30}{#1}}
\newcommand{\WarningTok}[1]{\textcolor[rgb]{0.37,0.37,0.37}{\textit{#1}}}

\providecommand{\tightlist}{%
  \setlength{\itemsep}{0pt}\setlength{\parskip}{0pt}}\usepackage{longtable,booktabs,array}
\usepackage{calc} % for calculating minipage widths
% Correct order of tables after \paragraph or \subparagraph
\usepackage{etoolbox}
\makeatletter
\patchcmd\longtable{\par}{\if@noskipsec\mbox{}\fi\par}{}{}
\makeatother
% Allow footnotes in longtable head/foot
\IfFileExists{footnotehyper.sty}{\usepackage{footnotehyper}}{\usepackage{footnote}}
\makesavenoteenv{longtable}
\usepackage{graphicx}
\makeatletter
\def\maxwidth{\ifdim\Gin@nat@width>\linewidth\linewidth\else\Gin@nat@width\fi}
\def\maxheight{\ifdim\Gin@nat@height>\textheight\textheight\else\Gin@nat@height\fi}
\makeatother
% Scale images if necessary, so that they will not overflow the page
% margins by default, and it is still possible to overwrite the defaults
% using explicit options in \includegraphics[width, height, ...]{}
\setkeys{Gin}{width=\maxwidth,height=\maxheight,keepaspectratio}
% Set default figure placement to htbp
\makeatletter
\def\fps@figure{htbp}
\makeatother

\KOMAoption{captions}{tableheading}
\makeatletter
\@ifpackageloaded{caption}{}{\usepackage{caption}}
\AtBeginDocument{%
\ifdefined\contentsname
  \renewcommand*\contentsname{Table of contents}
\else
  \newcommand\contentsname{Table of contents}
\fi
\ifdefined\listfigurename
  \renewcommand*\listfigurename{List of Figures}
\else
  \newcommand\listfigurename{List of Figures}
\fi
\ifdefined\listtablename
  \renewcommand*\listtablename{List of Tables}
\else
  \newcommand\listtablename{List of Tables}
\fi
\ifdefined\figurename
  \renewcommand*\figurename{Figure}
\else
  \newcommand\figurename{Figure}
\fi
\ifdefined\tablename
  \renewcommand*\tablename{Table}
\else
  \newcommand\tablename{Table}
\fi
}
\@ifpackageloaded{float}{}{\usepackage{float}}
\floatstyle{ruled}
\@ifundefined{c@chapter}{\newfloat{codelisting}{h}{lop}}{\newfloat{codelisting}{h}{lop}[chapter]}
\floatname{codelisting}{Listing}
\newcommand*\listoflistings{\listof{codelisting}{List of Listings}}
\makeatother
\makeatletter
\makeatother
\makeatletter
\@ifpackageloaded{caption}{}{\usepackage{caption}}
\@ifpackageloaded{subcaption}{}{\usepackage{subcaption}}
\makeatother

\ifLuaTeX
  \usepackage{selnolig}  % disable illegal ligatures
\fi
\usepackage{bookmark}

\IfFileExists{xurl.sty}{\usepackage{xurl}}{} % add URL line breaks if available
\urlstyle{same} % disable monospaced font for URLs
\hypersetup{
  pdftitle={Lab 2},
  colorlinks=true,
  linkcolor={blue},
  filecolor={Maroon},
  citecolor={Blue},
  urlcolor={Blue},
  pdfcreator={LaTeX via pandoc}}


\title{Lab 2}
\author{}
\date{}

\begin{document}
\maketitle


Welcome back to the lab again. From last week, you may have worked with
mass spectrometry data and processed a bit of them by yourself. This lab
will look into the results that you have make and we will practice a bit
of data wrangling and visualization with R. Thus, you can now understand
the dataset much better with many more aspects from your result. As
usual, please fill in your information here so we can give you nice some
nice scores later.

\begin{itemize}
\tightlist
\item
  Member1:
\item
  Member2:
\item
  Contact email:
\end{itemize}

Same here, there will be ten questions and three bonus questions for you
to answer. Please try to elaborate this exercise with the lectures from
the first weeks. The main goal of this lab is that you are not afraid to
work with mass spectrometry as it is amazing. Woo hoo.

\subsection{Intended learning outcomes
(ILOs)}\label{intended-learning-outcomes-ilos}

On completion of the lab, the student should be able to:

\begin{verbatim}
* demonstrate data-processing procedures in mass spectrometry proteomics
* demonstrate the ability to answer statistical questions with computational tools in mass spectrometry
* identify quality of high-throughput dataset and handle with statistical understandings
* identify relevant issues in technologies and data with accessible visualization techniques
\end{verbatim}

\subsection{Let's start!}\label{lets-start}

You may recall from what we have done in the first lab. Now, we want to
look at them properly. Let's start with some basic R programming skill.
So, please just copy the file and change the path below. (If you don't
rememer, just take it from your submission)

\begin{Shaded}
\begin{Highlighting}[]
\CommentTok{\# Libraries {-}{-}{-}{-} }
\CommentTok{\#| warning: false}
\CommentTok{\#| echo: false}
\CommentTok{\#| message: false}

\FunctionTok{library}\NormalTok{(dplyr)}
\end{Highlighting}
\end{Shaded}

\begin{verbatim}

Attaching package: 'dplyr'
\end{verbatim}

\begin{verbatim}
The following objects are masked from 'package:stats':

    filter, lag
\end{verbatim}

\begin{verbatim}
The following objects are masked from 'package:base':

    intersect, setdiff, setequal, union
\end{verbatim}

\begin{Shaded}
\begin{Highlighting}[]
\FunctionTok{library}\NormalTok{(ggplot2)}
\FunctionTok{library}\NormalTok{(tidyverse)}
\end{Highlighting}
\end{Shaded}

\begin{verbatim}
-- Attaching core tidyverse packages ------------------------ tidyverse 2.0.0 --
v forcats   1.0.0     v stringr   1.5.1
v lubridate 1.9.3     v tibble    3.2.1
v purrr     1.0.2     v tidyr     1.3.1
v readr     2.1.5     
\end{verbatim}

\begin{verbatim}
-- Conflicts ------------------------------------------ tidyverse_conflicts() --
x dplyr::filter() masks stats::filter()
x dplyr::lag()    masks stats::lag()
i Use the conflicted package (<http://conflicted.r-lib.org/>) to force all conflicts to become errors
\end{verbatim}

\begin{Shaded}
\begin{Highlighting}[]
\FunctionTok{library}\NormalTok{(stringr)}
\FunctionTok{library}\NormalTok{(visdat)}
\FunctionTok{library}\NormalTok{(naniar)}



\CommentTok{\# File path of the result}
\CommentTok{\# Fix when export}
\NormalTok{path }\OtherTok{\textless{}{-}} \StringTok{\textquotesingle{}/home/thanadol/Documents/GitHub/CB2110/lab2/sdrf\_openms\_design\_msstats\_in.csv\textquotesingle{}}
\NormalTok{ms\_result }\OtherTok{\textless{}{-}} \FunctionTok{read.csv}\NormalTok{(path)}
\end{Highlighting}
\end{Shaded}

\section{Data overview}\label{data-overview}

Let's check how do your data look. You can do it basically with any
spreadsheet or texteditor software. R is also one of them so don't be
afraid.

\begin{Shaded}
\begin{Highlighting}[]
\FunctionTok{head}\NormalTok{(ms\_result)}
\end{Highlighting}
\end{Shaded}

\begin{verbatim}
            ProteinName        PeptideSequence PrecursorCharge FragmentIon
1  sp|Q14974|IMB1_HUMAN     .(Acetyl)MELITILEK               2          NA
2 sp|P55011|S12A2_HUMAN AAAAAAAAAAAAAAAGAGAGAK               2          NA
3 sp|P55011|S12A2_HUMAN AAAAAAAAAAAAAAAGAGAGAK               2          NA
4 sp|P55011|S12A2_HUMAN AAAAAAAAAAAAAAAGAGAGAK               2          NA
5 sp|P55011|S12A2_HUMAN AAAAAAAAAAAAAAAGAGAGAK               3          NA
6  sp|P07954|FUMH_HUMAN         AAAEVNQDYGLDPK               2          NA
  ProductCharge IsotopeLabelType                          Condition
1             0                L                             normal
2             0                L colorectal cancer liver metastases
3             0                L colorectal cancer liver metastases
4             0                L colorectal cancer liver metastases
5             0                L colorectal cancer liver metastases
6             0                L colorectal cancer liver metastases
  BioReplicate Run Intensity                              Reference
1            2   4  11335670 140521_MO_H710_Tn8_1_140523083511.mzML
2            1   1  71453700              140521_MO_H710_TP2_1.mzML
3            1   2  79248010              140521_MO_H710_TP2_2.mzML
4            1   3  61905350              140521_MO_H710_TP2_3.mzML
5            1   3  43845410              140521_MO_H710_TP2_3.mzML
6            1   2  11507960              140521_MO_H710_TP2_2.mzML
\end{verbatim}

\begin{Shaded}
\begin{Highlighting}[]
\CommentTok{\# or }
\CommentTok{\# View(ms\_result)}
\end{Highlighting}
\end{Shaded}

\subsection{Q1.}\label{q1.}

\textbf{What do you see in the output file? What are the columns and the
rows}

Ans.

You can now see that it's difficult to see them clearly as there are
many rows. It's hard to get an overview of the dataset. Let's dig a bit
deeper and make it a little more organized. Let's now check numbers of
samples. These should be related to what you have with the SDRF file
previously.

\begin{Shaded}
\begin{Highlighting}[]
\CommentTok{\# Number of samples and what are they}
\NormalTok{ms\_result }\SpecialCharTok{\%\textgreater{}\%} 
  \CommentTok{\# Select column Reference and Condition}
  \FunctionTok{select}\NormalTok{(Reference, Condition) }\SpecialCharTok{\%\textgreater{}\%} 
  \CommentTok{\# Remove duplicated rows in the table}
  \FunctionTok{distinct}\NormalTok{()}
\end{Highlighting}
\end{Shaded}

\begin{verbatim}
                               Reference                          Condition
1 140521_MO_H710_Tn8_1_140523083511.mzML                             normal
2              140521_MO_H710_TP2_1.mzML colorectal cancer liver metastases
3              140521_MO_H710_TP2_2.mzML colorectal cancer liver metastases
4              140521_MO_H710_TP2_3.mzML colorectal cancer liver metastases
5              140521_MO_H710_Tn8_2.mzML                             normal
6              140521_MO_H710_Tn8_3.mzML                             normal
\end{verbatim}

Does it look similar? Definitely, haha.

Now, look at the proteins.

\subsection{Q2.}\label{q2.}

\textbf{Let's check the peptides and proteins. How many unique proteins
and peptides? What percentaged of matched protein from your library?}

\begin{Shaded}
\begin{Highlighting}[]
\CommentTok{\# Peptides and Protein }
\CommentTok{\# Extract a table of three columns with unique proteins, peptides and precursor charge}

\CommentTok{\# ms\_result \%\textgreater{}\% }
\CommentTok{\#   select(\_,\_,\_) }
\CommentTok{\#   distinct(\_\_\_)}
\end{Highlighting}
\end{Shaded}

Ans.

\subsection{Q3.}\label{q3.}

\textbf{Do you see any modification during sample pereparation? If yes,
why do we need them?}

Ans.

\subsection{BQ1.}\label{bq1.}

\textbf{Please show a summarised table containing numbers of protein
counts and the detectable peptide numbers. For example, there are 10
proteins and each of them has 5 detectable peptides.}

\begin{Shaded}
\begin{Highlighting}[]
\CommentTok{\# Peptide per protein count}
\CommentTok{\# Hint: count()}
\CommentTok{\# This will give you numbers of duplicated items in a particular vector column}
\CommentTok{\# Fix the code below}
\NormalTok{ms\_result }\SpecialCharTok{\%\textgreater{}\%} 
  \FunctionTok{select}\NormalTok{(ProteinName, PeptideSequence) }\SpecialCharTok{\%\textgreater{}\%}
  \FunctionTok{distinct}\NormalTok{() }\SpecialCharTok{\%\textgreater{}\%} 
  \FunctionTok{count}\NormalTok{(ProteinName) }\SpecialCharTok{\%\textgreater{}\%} 
  \FunctionTok{arrange}\NormalTok{(n) }\SpecialCharTok{\%\textgreater{}\%} \FunctionTok{count}\NormalTok{(n)}
\end{Highlighting}
\end{Shaded}

\begin{verbatim}
Storing counts in `nn`, as `n` already present in input
i Use `name = "new_name"` to pick a new name.
\end{verbatim}

\begin{verbatim}
    n  nn
1   1 491
2   2 263
3   3 149
4   4  91
5   5  50
6   6  46
7   7  22
8   8  23
9   9  15
10 10   8
11 11   6
12 12   6
13 13   5
14 14   2
15 15   5
16 16   2
17 18   1
18 20   1
19 23   1
20 24   1
21 25   1
22 27   1
23 30   1
24 33   1
25 34   1
\end{verbatim}

\section{Dynamic range}\label{dynamic-range}

Now, let's roughly look at the intensity of the peptides. We will use
ggplot2 to visualize the data.

\begin{Shaded}
\begin{Highlighting}[]
\NormalTok{ms\_result }\SpecialCharTok{\%\textgreater{}\%} 
  \CommentTok{\# Cobcatenate PeptideSequence and PrecursorCharge to new column}
  \FunctionTok{mutate}\NormalTok{(}\AttributeTok{PeptideSequencePC =} \FunctionTok{str\_c}\NormalTok{(PeptideSequence, }\StringTok{"\_"}\NormalTok{, PrecursorCharge)) }\SpecialCharTok{\%\textgreater{}\%}
  \CommentTok{\# plot the all peptide and Precursor and }
  \CommentTok{\# color by Condition}
  \FunctionTok{ggplot}\NormalTok{(}\FunctionTok{aes}\NormalTok{(}\AttributeTok{x =}\FunctionTok{reorder}\NormalTok{(PeptideSequencePC, Intensity), }
  \AttributeTok{y =}\NormalTok{ Intensity, }\AttributeTok{color =}\NormalTok{ Condition)) }\SpecialCharTok{+}
  \FunctionTok{geom\_point}\NormalTok{(}\AttributeTok{alpha =}\NormalTok{ .}\DecValTok{4}\NormalTok{, }\FunctionTok{aes}\NormalTok{(}\AttributeTok{color =}\NormalTok{ Reference)) }\SpecialCharTok{+} 
  \FunctionTok{geom\_smooth}\NormalTok{(}\FunctionTok{aes}\NormalTok{(}\AttributeTok{group =}\NormalTok{ Reference)) }\SpecialCharTok{+}
  \FunctionTok{labs}\NormalTok{(}\AttributeTok{x =} \StringTok{"Peptides"}\NormalTok{, }\AttributeTok{y =} \StringTok{"Intensity"}\NormalTok{,}
  \AttributeTok{title =} \StringTok{"Overall peptide Intensity"}\NormalTok{)}\SpecialCharTok{+} 
  \FunctionTok{scale\_y\_log10}\NormalTok{(}\AttributeTok{trans =} \StringTok{\textquotesingle{}log10\textquotesingle{}}\NormalTok{, }\AttributeTok{breaks =} \DecValTok{10}\SpecialCharTok{\^{}}\NormalTok{(}\DecValTok{7}\SpecialCharTok{:}\DecValTok{12}\NormalTok{)) }\SpecialCharTok{+} 
  \FunctionTok{theme\_classic}\NormalTok{() }\SpecialCharTok{+}
  \FunctionTok{theme}\NormalTok{(}\AttributeTok{legend.position =} \StringTok{"bottom"}\NormalTok{,}
  \AttributeTok{axis.text.x =} \FunctionTok{element\_blank}\NormalTok{(),}
  \AttributeTok{axis.ticks.x =} \FunctionTok{element\_blank}\NormalTok{(),}
  \AttributeTok{plot.title =} \FunctionTok{element\_text}\NormalTok{(}\AttributeTok{hjust =} \FloatTok{0.5}\NormalTok{)) }
\end{Highlighting}
\end{Shaded}

\begin{verbatim}
`geom_smooth()` using method = 'gam' and formula = 'y ~ s(x, bs = "cs")'
\end{verbatim}

\pandocbounded{\includegraphics[keepaspectratio]{lab2_files/figure-pdf/unnamed-chunk-6-1.pdf}}

You now can see the dynamic range of every proteins that were detected.
The intensity of the peptides is quite different maybe in some samples.

\subsection{Q4.}\label{q4.}

\textbf{What can we imply from this plot? What is the dynamic range of
the dataset?}

Ans.

\subsection{Q5.}\label{q5.}

\textbf{Pick one protein that is comprised of 10 detectable peptides.
Visulize the peptide intensity. Do we see every peptide in every sample?
Does every peptide have the same intensity in each sample for each
peptide. If not, why?}

\begin{Shaded}
\begin{Highlighting}[]
\CommentTok{\# Select a protein with 10 peptides that are detected in every sample}
\CommentTok{\# Fix the code below}

\NormalTok{prot\_10 }\OtherTok{\textless{}{-}}\NormalTok{ ms\_result }\SpecialCharTok{\%\textgreater{}\%} 
  \FunctionTok{select}\NormalTok{(ProteinName, PeptideSequence) }\SpecialCharTok{\%\textgreater{}\%}
  \FunctionTok{distinct}\NormalTok{() }\SpecialCharTok{\%\textgreater{}\%}
  \FunctionTok{count}\NormalTok{(ProteinName) }\SpecialCharTok{\%\textgreater{}\%}
  \FunctionTok{filter}\NormalTok{(n }\SpecialCharTok{==} \DecValTok{10}\NormalTok{) }\SpecialCharTok{\%\textgreater{}\%}
  \FunctionTok{slice\_sample}\NormalTok{(}\AttributeTok{n =} \DecValTok{1}\NormalTok{) }\SpecialCharTok{\%\textgreater{}\%} 
  \FunctionTok{pull}\NormalTok{(ProteinName)}

\CommentTok{\# Print out the protein name}

\FunctionTok{print}\NormalTok{(prot\_10)}
\end{Highlighting}
\end{Shaded}

\begin{verbatim}
[1] "sp|Q9Y6C2|EMIL1_HUMAN"
\end{verbatim}

\begin{Shaded}
\begin{Highlighting}[]
\CommentTok{\# Vislualize it }

\NormalTok{ms\_result }\SpecialCharTok{\%\textgreater{}\%} 
  \FunctionTok{filter}\NormalTok{(ProteinName }\SpecialCharTok{==}\NormalTok{ prot\_10) }\SpecialCharTok{\%\textgreater{}\%}  
  \FunctionTok{ggplot}\NormalTok{(}\FunctionTok{aes}\NormalTok{(}\AttributeTok{x =}\FunctionTok{reorder}\NormalTok{(PeptideSequence, Intensity, mean), }\AttributeTok{y =}\NormalTok{ Intensity, }\AttributeTok{color =}\NormalTok{ Reference, }\AttributeTok{group=}\NormalTok{Reference)) }\SpecialCharTok{+}
  \FunctionTok{facet\_wrap}\NormalTok{(}\SpecialCharTok{\textasciitilde{}}\NormalTok{PrecursorCharge) }\SpecialCharTok{+}
  \FunctionTok{geom\_point}\NormalTok{(}\AttributeTok{alpha =}\NormalTok{ .}\DecValTok{4}\NormalTok{) }\SpecialCharTok{+} \FunctionTok{geom\_line}\NormalTok{() }\SpecialCharTok{+}
  \FunctionTok{theme\_classic}\NormalTok{() }
\end{Highlighting}
\end{Shaded}

\pandocbounded{\includegraphics[keepaspectratio]{lab2_files/figure-pdf/unnamed-chunk-7-1.pdf}}

\#Ans.

\section{Missing data}\label{missing-data}

Let's now visualize the data that you already have with their
intensities. We will look the data at the peptide precursor level.

\begin{Shaded}
\begin{Highlighting}[]
\NormalTok{select\_pept }\OtherTok{\textless{}{-}}\NormalTok{ ms\_result }\SpecialCharTok{\%\textgreater{}\%} 
  \FunctionTok{select}\NormalTok{(Reference, PeptideSequence, PrecursorCharge, Intensity) }\SpecialCharTok{\%\textgreater{}\%} 
  \FunctionTok{pivot\_wider}\NormalTok{(}\AttributeTok{names\_from =}\NormalTok{ Reference, }\AttributeTok{values\_from =}\NormalTok{ Intensity) }


\FunctionTok{vis\_miss}\NormalTok{(select\_pept, }\AttributeTok{cluster =} \ConstantTok{TRUE}\NormalTok{)}
\end{Highlighting}
\end{Shaded}

\pandocbounded{\includegraphics[keepaspectratio]{lab2_files/figure-pdf/unnamed-chunk-8-1.pdf}}

The problem now is that we can detect some missing data in the dataset.
This is a common problem in mass spectrometry data. We can see that some
peptides are not detected in some samples.

\subsection{Q6.}\label{q6.}

\textbf{Why there are missing values with MS? Summarise the list of
unique peptide to only one protein}

Ans.

Let's now select good quality peptide based on the missing data. We will
remove the peptides that are not detected in more than 50\% of the
samples.

\subsection{Q7.}\label{q7.}

\textbf{From \texttt{select\_pept} table, remove the peptides that are
not detected in more than 50\% of the samples. How many protiens and
peptides are left?}

\begin{Shaded}
\begin{Highlighting}[]
\CommentTok{\# Remove the peptides that are not detected in more than 50\% of the samples}
\CommentTok{\# Hint: pivot\_longer() and filter()}
\CommentTok{\# Fix the code below}

\NormalTok{filter\_pept }\OtherTok{\textless{}{-}}\NormalTok{ select\_pept }\SpecialCharTok{\%\textgreater{}\%} 
  \FunctionTok{pivot\_longer}\NormalTok{(}\AttributeTok{cols =} \SpecialCharTok{{-}}\FunctionTok{c}\NormalTok{(PeptideSequence, PrecursorCharge), }\AttributeTok{names\_to =} \StringTok{"Reference"}\NormalTok{, }\AttributeTok{values\_to =} \StringTok{"Intensity"}\NormalTok{) }\SpecialCharTok{\%\textgreater{}\%} 
  \FunctionTok{group\_by}\NormalTok{(PeptideSequence, PrecursorCharge) }\SpecialCharTok{\%\textgreater{}\%}
  \FunctionTok{summarise}\NormalTok{(}\AttributeTok{n\_na =} \FunctionTok{sum}\NormalTok{(}\FunctionTok{is.na}\NormalTok{(Intensity))) }\SpecialCharTok{\%\textgreater{}\%}
  \FunctionTok{filter}\NormalTok{(n\_na }\SpecialCharTok{\textless{}} \DecValTok{3}\NormalTok{) }\SpecialCharTok{\%\textgreater{}\%} 
  \FunctionTok{select}\NormalTok{(PeptideSequence, PrecursorCharge)}
\end{Highlighting}
\end{Shaded}

\begin{verbatim}
`summarise()` has grouped output by 'PeptideSequence'. You can override using
the `.groups` argument.
\end{verbatim}

\begin{Shaded}
\begin{Highlighting}[]
\NormalTok{ms\_result\_filt }\OtherTok{\textless{}{-}}\NormalTok{ ms\_result }\SpecialCharTok{\%\textgreater{}\%}
  \FunctionTok{inner\_join}\NormalTok{(filter\_pept, }\AttributeTok{by =} \FunctionTok{c}\NormalTok{(}\StringTok{"PeptideSequence"}\NormalTok{, }\StringTok{"PrecursorCharge"}\NormalTok{)) }

\CommentTok{\# Protein count}
\NormalTok{ms\_result\_filt }\SpecialCharTok{\%\textgreater{}\%} 
  \FunctionTok{select}\NormalTok{(ProteinName, PeptideSequence) }\SpecialCharTok{\%\textgreater{}\%}
  \FunctionTok{distinct}\NormalTok{() }\SpecialCharTok{\%\textgreater{}\%}
  \FunctionTok{count}\NormalTok{(ProteinName) }\SpecialCharTok{\%\textgreater{}\%} 
  \FunctionTok{count}\NormalTok{(n)}
\end{Highlighting}
\end{Shaded}

\begin{verbatim}
Storing counts in `nn`, as `n` already present in input
i Use `name = "new_name"` to pick a new name.
\end{verbatim}

\begin{verbatim}
    n  nn
1   1 228
2   2  91
3   3  38
4   4  25
5   5  12
6   6  11
7   7   3
8   8   3
9   9   2
10 10   2
11 11   1
12 20   2
\end{verbatim}

\begin{Shaded}
\begin{Highlighting}[]
\CommentTok{\# Peptide count}
\NormalTok{ms\_result\_filt }\SpecialCharTok{\%\textgreater{}\%} 
  \FunctionTok{select}\NormalTok{(ProteinName, PeptideSequence) }\SpecialCharTok{\%\textgreater{}\%}
  \FunctionTok{distinct}\NormalTok{() }\SpecialCharTok{\%\textgreater{}\%}
  \FunctionTok{count}\NormalTok{(PeptideSequence) }\SpecialCharTok{\%\textgreater{}\%} 
  \FunctionTok{count}\NormalTok{(n)}
\end{Highlighting}
\end{Shaded}

\begin{verbatim}
Storing counts in `nn`, as `n` already present in input
i Use `name = "new_name"` to pick a new name.
\end{verbatim}

\begin{verbatim}
  n  nn
1 1 884
\end{verbatim}

Looks like we are more confident with the data now. Let's now calculate
the missing data in the dataset. The rule is we expect all peptide
precursors to be detected with similar intensity in every protein.
Meaning that, we can average the intensity of the peptides in each
protein and compare them with the rest.

\subsection{Q8.}\label{q8.}

\textbf{Calculate the missing data in the dataset. What is the protein
concentration in each sample}

\begin{Shaded}
\begin{Highlighting}[]
\CommentTok{\# Hint: \# group\_by() and summarise()}
\CommentTok{\# Fix the code below}

\NormalTok{prot\_level }\OtherTok{\textless{}{-}}\NormalTok{ ms\_result\_filt }\SpecialCharTok{\%\textgreater{}\%} 
  \FunctionTok{group\_by}\NormalTok{(ProteinName, Reference) }\SpecialCharTok{\%\textgreater{}\%} 
  \FunctionTok{summarise}\NormalTok{(}\AttributeTok{mean\_intensity =} \FunctionTok{mean}\NormalTok{(Intensity, }\AttributeTok{na.rm =} \ConstantTok{TRUE}\NormalTok{)) }\SpecialCharTok{\%\textgreater{}\%} 
  \FunctionTok{ungroup}\NormalTok{()}
\end{Highlighting}
\end{Shaded}

\begin{verbatim}
`summarise()` has grouped output by 'ProteinName'. You can override using the
`.groups` argument.
\end{verbatim}

\section{Protein concentration}\label{protein-concentration}

\subsection{Q9.}\label{q9.}

\textbf{Let's plot a dynamic range of protein concentration in each
sample. What can we imply from this plot?}

\begin{Shaded}
\begin{Highlighting}[]
\NormalTok{prot\_level }\SpecialCharTok{\%\textgreater{}\%} 
   \FunctionTok{left\_join}\NormalTok{(ms\_result }\SpecialCharTok{\%\textgreater{}\%} \FunctionTok{select}\NormalTok{(Reference, Condition) }\SpecialCharTok{\%\textgreater{}\%} \FunctionTok{distinct}\NormalTok{(), }\AttributeTok{by =} \StringTok{"Reference"}\NormalTok{) }\SpecialCharTok{\%\textgreater{}\%}
   \FunctionTok{ggplot}\NormalTok{(}\FunctionTok{aes}\NormalTok{(}\AttributeTok{x =} \FunctionTok{reorder}\NormalTok{(ProteinName, mean\_intensity), }\AttributeTok{y =}\NormalTok{ mean\_intensity, }\AttributeTok{color=}\NormalTok{ Reference)) }\SpecialCharTok{+}
   \FunctionTok{geom\_point}\NormalTok{(}\AttributeTok{alpha =}\NormalTok{ .}\DecValTok{4}\NormalTok{) }\SpecialCharTok{+}  
  \FunctionTok{geom\_smooth}\NormalTok{(}\FunctionTok{aes}\NormalTok{(}\AttributeTok{group =}\NormalTok{ Condition, }\AttributeTok{color =}\NormalTok{ Condition)) }\SpecialCharTok{+}
    \FunctionTok{scale\_y\_log10}\NormalTok{() }\SpecialCharTok{+} 
  \FunctionTok{labs}\NormalTok{(}\AttributeTok{x =} \StringTok{"Proteins"}\NormalTok{, }\AttributeTok{y =} \StringTok{"Intensity"}\NormalTok{,}
  \AttributeTok{title =} \StringTok{"Overall protein Intensity"}\NormalTok{)}\SpecialCharTok{+} 
  \FunctionTok{theme\_classic}\NormalTok{() }\SpecialCharTok{+}
  \FunctionTok{theme}\NormalTok{(}\AttributeTok{legend.position =} \StringTok{"bottom"}\NormalTok{,}
  \AttributeTok{axis.text.x =} \FunctionTok{element\_blank}\NormalTok{(),}
  \AttributeTok{axis.ticks.x =} \FunctionTok{element\_blank}\NormalTok{(),}
  \AttributeTok{plot.title =} \FunctionTok{element\_text}\NormalTok{(}\AttributeTok{hjust =} \FloatTok{0.5}\NormalTok{))}
\end{Highlighting}
\end{Shaded}

\begin{verbatim}
`geom_smooth()` using method = 'gam' and formula = 'y ~ s(x, bs = "cs")'
\end{verbatim}

\pandocbounded{\includegraphics[keepaspectratio]{lab2_files/figure-pdf/unnamed-chunk-11-1.pdf}}

\subsection{Q10.}\label{q10.}

\textbf{Visualize the concentration of protein from PSMB3 gene with
boxplot. What can we summarise here? What shall we find next?}

\begin{Shaded}
\begin{Highlighting}[]
\CommentTok{\# hint}
\CommentTok{\# Fix this code below}

\CommentTok{\# Select Protein}
\NormalTok{prot }\OtherTok{\textless{}{-}}\NormalTok{ prot\_level }\SpecialCharTok{\%\textgreater{}\%} 
    \FunctionTok{left\_join}\NormalTok{(ms\_result }\SpecialCharTok{\%\textgreater{}\%} \FunctionTok{select}\NormalTok{(Reference, Condition) }\SpecialCharTok{\%\textgreater{}\%} \FunctionTok{distinct}\NormalTok{(), }\AttributeTok{by =} \StringTok{"Reference"}\NormalTok{) }\SpecialCharTok{\%\textgreater{}\%}
    \FunctionTok{select}\NormalTok{(ProteinName, Condition, mean\_intensity) }\SpecialCharTok{\%\textgreater{}\%}
    \FunctionTok{filter}\NormalTok{(ProteinName }\SpecialCharTok{==} \StringTok{\textquotesingle{}sp|O00264|PGRC1\_HUMAN\textquotesingle{}}\NormalTok{)}
\CommentTok{\# \# Boxplot}
\NormalTok{prot }\SpecialCharTok{\%\textgreater{}\%} 
  \FunctionTok{ggplot}\NormalTok{(}\FunctionTok{aes}\NormalTok{(}\AttributeTok{x =}\NormalTok{ Condition, }\AttributeTok{y =}\NormalTok{ mean\_intensity, }\AttributeTok{fill =}\NormalTok{ Condition)) }\SpecialCharTok{+} 
  \FunctionTok{geom\_boxplot}\NormalTok{() }\SpecialCharTok{+} 
  \FunctionTok{geom\_point}\NormalTok{() }\SpecialCharTok{+} 
  \FunctionTok{theme\_classic}\NormalTok{()}
\end{Highlighting}
\end{Shaded}

\pandocbounded{\includegraphics[keepaspectratio]{lab2_files/figure-pdf/unnamed-chunk-12-1.pdf}}

\begin{Shaded}
\begin{Highlighting}[]
\CommentTok{\# prot}
\end{Highlighting}
\end{Shaded}

\subsection{BQ2.}\label{bq2.}

\textbf{Plot the concentration of the most highly differentiated
protein}

\begin{Shaded}
\begin{Highlighting}[]
\NormalTok{prot\_max }\OtherTok{\textless{}{-}}\NormalTok{ prot\_level }\SpecialCharTok{\%\textgreater{}\%} 
    \FunctionTok{left\_join}\NormalTok{(ms\_result }\SpecialCharTok{\%\textgreater{}\%} \FunctionTok{select}\NormalTok{(Reference, Condition) }\SpecialCharTok{\%\textgreater{}\%} \FunctionTok{distinct}\NormalTok{(), }\AttributeTok{by =} \StringTok{"Reference"}\NormalTok{) }\SpecialCharTok{\%\textgreater{}\%}
    \FunctionTok{select}\NormalTok{(ProteinName, Condition, mean\_intensity) }\SpecialCharTok{\%\textgreater{}\%}
    \FunctionTok{group\_by}\NormalTok{(ProteinName, Condition) }\SpecialCharTok{\%\textgreater{}\%} 
    \FunctionTok{summarise}\NormalTok{(}\AttributeTok{mean\_intensity =} \FunctionTok{mean}\NormalTok{(mean\_intensity)) }\SpecialCharTok{\%\textgreater{}\%} 
    \FunctionTok{ungroup}\NormalTok{() }\SpecialCharTok{\%\textgreater{}\%}
    \FunctionTok{pivot\_wider}\NormalTok{(}\AttributeTok{names\_from =}\NormalTok{ Condition, }\AttributeTok{values\_from =}\NormalTok{ mean\_intensity) }\SpecialCharTok{\%\textgreater{}\%} 
    \FunctionTok{rename}\NormalTok{(}\AttributeTok{cancer =} \StringTok{\textasciigrave{}}\AttributeTok{colorectal cancer liver metastases}\StringTok{\textasciigrave{}}\NormalTok{) }\SpecialCharTok{\%\textgreater{}\%} 
    \FunctionTok{mutate}\NormalTok{(}\AttributeTok{diff =} \FunctionTok{abs}\NormalTok{(normal}\SpecialCharTok{{-}}\NormalTok{cancer)) }\SpecialCharTok{\%\textgreater{}\%}
    \FunctionTok{filter}\NormalTok{(diff }\SpecialCharTok{==} \FunctionTok{max}\NormalTok{(diff)) }\SpecialCharTok{\%\textgreater{}\%}
    \FunctionTok{pull}\NormalTok{(ProteinName)}
\end{Highlighting}
\end{Shaded}

\begin{verbatim}
`summarise()` has grouped output by 'ProteinName'. You can override using the
`.groups` argument.
\end{verbatim}

\begin{Shaded}
\begin{Highlighting}[]
\FunctionTok{print}\NormalTok{(prot\_max)}
\end{Highlighting}
\end{Shaded}

\begin{verbatim}
[1] "sp|P62805|H4_HUMAN"
\end{verbatim}

\begin{Shaded}
\begin{Highlighting}[]
\NormalTok{prot\_max }\OtherTok{\textless{}{-}}\NormalTok{ prot\_level }\SpecialCharTok{\%\textgreater{}\%} 
    \FunctionTok{left\_join}\NormalTok{(ms\_result }\SpecialCharTok{\%\textgreater{}\%} \FunctionTok{select}\NormalTok{(Reference, Condition) }\SpecialCharTok{\%\textgreater{}\%} \FunctionTok{distinct}\NormalTok{(), }\AttributeTok{by =} \StringTok{"Reference"}\NormalTok{) }\SpecialCharTok{\%\textgreater{}\%}
    \FunctionTok{select}\NormalTok{(ProteinName, Condition, mean\_intensity) }\SpecialCharTok{\%\textgreater{}\%}
    \FunctionTok{filter}\NormalTok{(ProteinName }\SpecialCharTok{==}\NormalTok{ prot\_max)}

\CommentTok{\# \# Boxplot}
\NormalTok{prot\_max }\SpecialCharTok{\%\textgreater{}\%} 
  \FunctionTok{ggplot}\NormalTok{(}\FunctionTok{aes}\NormalTok{(}\AttributeTok{x =}\NormalTok{ Condition, }\AttributeTok{y =}\NormalTok{ mean\_intensity, }\AttributeTok{fill =}\NormalTok{ Condition)) }\SpecialCharTok{+} 
  \FunctionTok{geom\_boxplot}\NormalTok{() }\SpecialCharTok{+} 
  \FunctionTok{geom\_point}\NormalTok{() }\SpecialCharTok{+} 
  \FunctionTok{theme\_classic}\NormalTok{() }
\end{Highlighting}
\end{Shaded}

\pandocbounded{\includegraphics[keepaspectratio]{lab2_files/figure-pdf/unnamed-chunk-13-1.pdf}}

\subsection{BQ3.}\label{bq3.}

\textbf{What are the advanageous and dis advantageous of MS acquition?}




\end{document}
